\chapter{PC algorithm}

\section{Estimating high-dimensional directed acyclic graphs with the PC-algorithm}

	\subsection{Introduction}
	Graphical models are a popular probabilistic tool to analyse and visualize conditional independence relationships between random variables.
	The major building blocks of these models are nodes (the random variables) and edges (conditional dependence).
	The structure of conditional independence among the random variables can be explored using the Markov properties.
	The estimation of a DAG from data is difficult due to the enormous size of the space of DAGs.
	The PC-algorithm is an alternative to greedy or structurally restricted approaches.
	It starts from a complete, undirected graph and deletes recursively edges based on conditional independence decisions.
	This yields an undirected graph that can be partially directed and further extended to represent the underlying DAG.
	This algorithm runs in the worst case in exponential time, but if the true underlying DAG is sparse, it is reduced to a polynomial runtime.
	Here there is a focus on estimating the equivalence class and the skeleton of DAGs corresopnding to multivariate Gaussian distributions in high-dimensional context, or where the number of nodes $p$ may be much larger than the sample size $n$.

	\subsection{Finding the equivalence class of a DAG}
	Let $G=(V,E)$ a graph.
	The set of vertices $V$ corresponds to the components of a random vector $\vec{X}\in\mathbb{R}^p$.
	A probability distribution $P$ on $\mathbb{R}^p$ is said to be faithful with respect to $G$ if conditional independences of the distribution can be inferred from $d$-separation in $G$.
	Consider a random vector $\vec{X}\sim P$.
	Faithfulness of $P$ with respect to $G$ means that for any $i,j\in V$ with $i\neq j$ and any set $s\subseteq V$:

	$$\vec{X}^{(i)}\land \vec{X}^{(j)}\text{ are conditionally independent given } \{\vec{X}^{(r)}|r\in s\}\Leftrightarrow i\land y \text{ are d-separated by } s$$

	Faithfulness is ruling out some classes of probability distributions.
	The skeleton of a $DAG$ is the undirected graph obtained from $G$ by substituting undirected edges for directed ones.
	A $v$-structure in a DAG is an ordered triple of node such that $(i,j)\in G\land (k,j)\in G\land (i,k)\not\in G$.
	Two DAGs are equivalent is and only is they have the same skeleton and v-structures.
	If $P$ is faithful with respect to a DAG $G$ there is an edge between node $i$ and $j$ in the skeleton of DAG $G$ is and only if $\forall s\subseteq V\backslash\{i,j\}, \vec{X}^{(i)}\land \vec{X}^{(j)}$ are conditionally dependent given $\{\vec{X}^{(r)},r\in s\}$.
	If $P$ is faithful with respect to a DAG $G$ the skeleton of the DAG is a subset to the conditional independence graph corresponding to $P$.
	Every edge in the skeleton indicates some strong dependence which cannot be explained by accounting for other variables.

		\subsubsection{PC-algorithm for finding the skeleton}
		The PC-algorithm betters from a naive strategies that checks for conditional independences given all subsets.

			\paragraph{Population version}
			In the population version of the PC-algorithm perfect knowledge about all necessary conditional independence relations is assumed available.

			\begin{algorithm}[H]
\DontPrintSemicolon
\SetKwComment{comment}{$\%$}{}
\SetKw{Int}{int}
\SetKw{To}{to}
\SetKw{Not}{not}
%\SetKw{Not}{not}
\SetKwData{Item}{item}
\SetKwFunction{Min}{min}
\SetKwFunction{PC}{$PC_{pop}$}

\caption{\protect\PC{$V$}}
l = -1\;
C = $\tilde{C}$\;
\Repeat{for each ordered pair of adjacend nodes $i,j:|adj(C,i)\backslash\{j\}|< l$}{
	l += 1\;
	\Repeat{all ordered pairs of adjacent variables $i$ and $j$ such that $|adj(C,i)\backslash\{j\}\ge l$ and $k\subseteq adj(C,i)\backslash\{j\}$ with $|k| = l$ have been tested for conditional independence}{
		Select a new ordered pair of nodes $i,j$ that are adjacent in $C$ such that $|adj(C,i)\backslash\{j\}|ge l$\;
		\Repeat{edge $i,j$ is deleted or all $k\subseteq adj(C,i)\backslash\{j\}$ with $|k| = l$ have been chosen}{
			Choose new $k\subseteq adj(C,i)\backslash\{j\}$ with $|k| = l$\;
			\If{$i\land j$ are conditionally independent given $k$}{
				Delete edge $i,j$\;
				Denote this new graph by $C$\;
				Save $k$ in $S(i,j)$ and $S(j,i)$\;
			}
		}
	}
}
\Return Estimated skeleton $C$, separation sets $S$\;
\end{algorithm}


			The maximal value of $l$ is denoted $m_{reach}$ and depends on the underlying distribution.
			Considering a DAG $G$ and assume that the distribution $P$ is faithful to $G$.
			Denote the maximal number of neighbours by $q = \max\limits_{1\le j\le p}|adj(G,j)|$.
			Then the $PC_{pop}$ algorithm construct the true skeleton of the DAG.
			Moreover $m_{reach}\in\{q-1,q\}$.

			\paragraph{Sample version}
			For finite sample there is a need to estimate conditional independencies.
			Assuming faithful models (conditional independence relations correspond to d-separations) in the Gaussian case conditional independences can be inferred from partial correlations.
			Assume that distribution $P$ of the random vector $\vec{X}$ is a multivariate normal.
			For $i\neq\in\{1,\dots,p\}, k\subseteq\{1,\dots, p\}\backslash\{i,j\}$, denote $\rho_{i,j|k}$ the partial correlation between $\vec{X}^{(i)}$ and $\vec{X}^{(j)}$ given $\{\vec{X}^{(r)}, r\in k\}$ then $\rho_{i,j|k} = 0$ if and only if $\vec{X}^{(i)}$ and $\vec{X}^{(j)}$ are conditionally independent given $\{\vec{X}^{(r)}, r\in k\}$.
			The partial correlations can be estimated and the sample partial correlation $\hat{\rho}_{i,j|k}$ can be calculated, for some $h\in k$:

			$$\rho_{i,j|k} = \frac{\rho_{i,j|k\backslash h}-\rho_{i,h|k\backslash h}\rho_{j,h|k\backslash h}}{\sqrt{(1-\rho^2_{i,h|k\backslash h})(1-\rho^2_{j,h|k\backslash h})}}$$

			For testing whether a partial correlation is zero or not Fisher's z-transform is applied:

			$$Z(i,j|k) = \frac{1}{2} \log(\frac{1=\hat{\rho}_{i,j|k}}{1-\hat{\rho}_{i,j|k}})$$

			The null hypothesis is rejected $H_o(i,j|k):\rho_{i,j|k} = 0$ against the two sided alternative $H_A(i,j|K): \rho_{i,j|k} \neq 0$ if $\sqrt{n-|k|-3}Z(i,j|k)>\Phi^{-1}(1-\frac{\alpha}{2})$, where $\Phi$ denotes the cdf of $\mathcal{N}(0,1)$.
			The sample version of the PC-algorithm is the same of the population version, with the if statement replaced by $\sqrt{n-|k|-3}Z(i,j|k)>\Phi^{-1}(1-\frac{\alpha}{2})$.
			This algorithm yields a data-dependent $\hat{m}_{reach,n}$.
			The only tuning parameter is $\alpha$, the significance level for testing partial correlations.
			This algorithm is asymptotically consistent even if $p$ is much larger than $n$ but the DAG is sparse.

		\subsubsection{Extending the skeleton to the equivalence class}
		While finding the skeleton the separations sets that made edges drop out were recorded by $S$.
		This is essential for extending the skeleton to the equivalence class.

			\begin{algorithm}[H]
\DontPrintSemicolon
\SetKwComment{comment}{$\%$}{}
\SetKw{Int}{int}
\SetKw{To}{to}
\SetKw{Not}{not}
\SetKw{All}{all}
\SetKw{RA}{R1}
\SetKw{RB}{R2}
\SetKw{RC}{R3}
\SetKw{RD}{R4}
\SetKwData{Item}{item}
\SetKwFunction{Min}{min}
\SetKwFunction{PC}{Extending\_to\_CPDAG}

\caption{\protect\PC{$G_{skel}, S$}}
\For{\All pairs of non adjacent $i,j$ with common neighbour $k$}{
	\If{$k\not\in S(i,j)$}{
		Replace $i-k-j\in G_{skel}$ with $i\rightarrow k\leftarrow j$\;
	}
}
In the resulting PDAG try to orient as many undirected edges as possible by repeated application of:\;
\RA orient $j-k$ into $j\rightarrow k$ whenever there is $i\rightarrow j$ such that $i$ and $k$ are non-adjacent.\;
\RB orient $i-j$ into $i\rightarrow j$ whenever there is a chain $i\rightarrow k\rightarrow j$.\;
\RC orient $i-j$ into $i\rightarrow j$ whenever there are two chains $i-k\rightarrow j$ and $i-l\rightarrow k$ such that $k$ and $l$ are non-adjacent.\;
\RD orient $i-j$ into $i\rightarrow j$ whenever there are two chains $i-k\rightarrow l$ and $k\rightarrow l\rightarrow j$ such that $k$ and $l$ are non-adjacent\;
\Return CPDAG $G$\;


\end{algorithm}


\section{A computing system for discovering causal relationships among human genes to improve drug repositioning}
The gene\@home project aims to expand gene networks using transcriptomic datasets.
For human data the objective is to provide a public resource to navigate and combine results by expanding each single human transcript.
The platform hosted the NE$S^2$RA algorithm.
Starting from a local gene network LGN based on previous biological knowledge, its expansion consists in a set of genes and a list of interactions which describe putative causal relationships with the genes in the LGN.
The expansion is calculated on observational gene expression data organized in a coherent normalized data matrix.
To overcome the problem of unique elaborations OneGenE has been developed.
The list of gene expansions is calculated for each gene in an organism by systematically running single-gene NE$S^2$RA expansions with fixed parameters and then combine them afterwards to simulate LGN expansions.
The expansions of the gene networks is based on the transcriptomic dataset provided by the FANTOM project.
Their data comes from sequencing of RNA extracted from different samples of human tissues and cell lines and contains expression profiles of $201802$ gene isoforms.
Drug repositioning is an alternative approach for the discovery of new therapeutic opportunities for already approved medicines.
This method, which relies on previous knowledge, speeds up the approval procedure of the drug regulators and can represent a valuable approach.

	\subsection{Method}
	OneGenE is a method to compute ranked candidate gene lists that expands known local gene networks given gene expression data.
	It is gbased on the systematic and iterative application of the skeleton function of the PC algorithm on subsets of the input data.
	Candidate expansion lists are pre-computed for each gene of the target organism.
	Secondly a set of transcript of interest LGN is provided as input and the intermediate results are aggregated.

		\subsubsection{Data and input}
		The algorithm starts with an $n\times m$ gene expression data matrix $E$, where $n=|S|$ is the number of transcripts $S$ and $m$ is the number of samples and a set of parametertuples $\Theta = \{\theta\} = \{(\alpha, d, i)|\alpha\in A, d\in D, i\in I\}$ where $A$, $D$ and $I$ are the sets of alpha values, tile sizes and the number of iterations.

		\subsubsection{Nested loop}
		For each transcript $g$ in $S$, $p$ instances of PC-IM are executed, where $p=|\Theta| = |A||D||I|$.
		The internal loop receive as input a single gene $g$ with probability vbector $\Pi = 1$.
		The internal loop comprises the subsetting of the transcripts, the run of PC-skeleton on each subset and the computation of absolute frequencies, relative frequencies and the corresponding order of the candidates list.6

		\subsubsection{Pre-computation result}
		In OneGenE the ranking aggregation is postponed.
		Each PC-IM returns a candidate expansion list $l_{g,\theta}$ for each tuple of parameters corresponding to the set of lists resulting from the algorithm.
		The candidate lists are stored and the data is ready to be queried.

		\subsubsection{Ranking or list aggregation}
		Let $S_{LGN}$ be the set of transcripts in an input LGN and $l_{g,\theta}$ the candidate expansion list of $g$ with parameters $\theta$.
		The final candidate expansion list is obtained combining $\mathcal{L} = \{l_{g,\theta}| g\in S_{LGN}, \theta\in\Theta\}$ by means of a ranking aggregator.
		High relative frequency provides evidence of a putative direct causal relationship.
		Ranks are instead useful for comparing list and prioritization.

	\subsection{Data and running paramters}
	The human transcriptome data have been obtained from the FANTOM5 project.
	It was generated using single molecule CAGE or cap analysis gene expressione.
	Normalized expression values are extimated as transcripts per milion TPM.
	The data has been filtered removing unknown transcripts and with absent or low expression values among almost all samples.
	The single gene NE$S^2$RA expansion where submitted with a tile size of $1000$, $1000$ iterations and $\alpha=0.05$.
