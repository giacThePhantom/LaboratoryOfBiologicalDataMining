\chapter{Introduction}
Data mining and data analysis are not the same thing: the former deals with data that pre-exists with respect to the research question, whereas the latter deals with data that is generated and collected in order to answer some research question.

\section{Type of data}
Data is canonically represented via tables.
It implies the existence of a mapping between facts of the world and symbols.
The measurement is a mapping between facts in the world and elements of sets equipped with some mathematical structure.

	\subsection{Categorical data}
	In categorical data the set is finite and it has no structure.
	The only operation that can be done on it is to distinguish the elements of the set.

	\subsection{Ordered categorical data}
	In ordered categorical data the set has an order: its element are in a mathematical relationship with the properties of an order:

	\begin{multicols}{3}
		\begin{itemize}
			\item Reflexivity.
			\item Antisymmetry.
			\item Transitivity.
		\end{itemize}
	\end{multicols}

	\subsection{Discrete data}
	In discrete data the set is a subset of the natural numbers.
	Operations permitted are sum and difference.

	\subsection{Continuous data}
	In continuous data the set is an interval of the real numbers.
	Some scales have an absolute zero.
	The operations permitted are sum and difference.
	Division makes sense only if the scale is absolute.

\section{Metadata}
Metadata is data about the data.
For example it contains informations about its origin, the method, time, format, source and owner.

\section{Interventional data and observational data}
Interventional data and observational data are distinguished on the basis of the intervention on the system.
The former is generated by experimental procedures and the latter by pure observation/
